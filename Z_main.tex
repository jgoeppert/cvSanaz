%% start of file `template.tex'.
%% Copyright 2006-2013 Xavier Danaux (xdanaux@gmail.com).
%
% This work may be distributed and/or modified under the
% conditions of the LaTeX Project Public License version 1.3c,
% available at http://www.latex-project.org/lppl/.

\documentclass[10pt,a4paper,sans]{moderncv}        % possible options include font size ('10pt', '11pt' and '12pt'), paper size ('a4paper', 'letterpaper', 'a5paper', 'legalpaper', 'executivepaper' and 'landscape') and font family ('sans' and 'roman')






% moderncv themes
% \moderncvstyle[left]{casual}                             % style options are 
\moderncvstyle{custom}                             % style options are 'casual' (default), 'classic', 'oldstyle' and 'banking'
\moderncvcolor{blue}                               % color options 'blue' (default), 'orange', 'green', 'red', 'purple', 'grey' and 'black'
%\renewcommand{\familydefault}{\sfdefault}         % to set the default font; use '\sfdefault' for the default sans serif font, '\rmdefault' for the default roman one, or any tex font name
%\nopagenumbers{}                                  % uncomment to suppress automatic page numbering for CVs longer than one page

\usepackage{ragged2e}


% character encoding
\usepackage[utf8]{inputenc}                       % if you are not using xelatex ou lualatex, replace by the encoding you are using
%\usepackage{CJKutf8}                              % if you need to use CJK to typeset your resume in Chinese, Japanese or Korean

% adjust the page margins
\usepackage[scale=0.8]{geometry}
%\setlength{\hintscolumnwidth}{3cm}                % if you want to change the width of the column with the dates
%\setlength{\makecvtitlenamewidth}{10cm}           % for the 'classic' style, if you want to force the width allocated to your name and avoid line breaks. be careful though, the length is normally calculated to avoid any overlap with your personal info; use this at your own typographical risks...

\usepackage[ngerman]{babel}
\usepackage[official]{eurosym}
\usepackage{emerald}
% \usepackage{arabtex}


% personal data
\name{Jacob}{Göppert}
\ifthenelse {\equal{\english}{\empty}}{\title{Lebenslauf}}{\title{Curriculum Vitae}}
\ifthenelse {\equal{\english}{\empty}}{\address{Merzhauser Stra\ss e 159F}{79100 Freiburg}{}}{\address{Merzhauser Stra\ss e 159F}{79100 Freiburg, Germany}{}}% optional, remove / comment the line if not wanted; the "postcode city" and and "country" arguments can be omitted or provided empty
\phone[mobile]{+49~(176)~238~13~212}                   % optional, remove / comment the line if not wanted
% \email{sanaz.a89@yahoo.com}                               % optional, remove / comment the line if not wanted
% \email{\normalsize{sanaz.goeppert-asadollahpour@web.de}}                               % optional, remove / comment the line if not wanted
% \email{sanaz.goeppert-asadollahpour@web.de}                               % optional, remove / comment the 
\email{jacob.goeppert@gmail.com}                               % optional, remove / comment the line if not wanted
% \email{sanaz.asadollahpour@uni-freiburg.de}                               % optional, remove / comment the 
% \email{sanaz.asadollahpour@ocbc.uni-freiburg.de}                               % optional, remove / comment the 

%\homepage{www.johndoe.com}                         % optional, remove / comment the line if not wanted
%\extrainfo{additional information}                 % optional, remove / comment the line if not wanted
\photo[6cm][0.25cm]{Seehstern_7797}                       % optional, remove / comment the line if not wanted; '64pt' is the height the picture must be resized to, 0.4pt is the thickness of the frame around it (put it to 0pt for no frame) and 'picture' is the name of the picture file
% \photo[5cm][0.25cm]{Jacob-pic}                       % optional, remove / comment the line if not wanted; '64pt' is the height the picture must be resized to, 0.4pt is the thickness of the frame around it (put it to 0pt for no frame) and 'picture' is the name of the picture file
% \photo[64pt][0.4pt]{Jacob-pic}                       % optional, remove / comment the line if not wanted; '64pt' is the height the picture must be resized to, 0.4pt is the thickness of the frame around it (put it to 0pt for no frame) and 'picture' is the name of the picture file

%\stelle{Chemikerin als Manager von Analytik-Dienstleistungen}
% \stelle{Biochemiker- PCR Assays (m/w/d) Referenznummer: DES00220220407}
% \stelle{Biochemiker als Sales Consultant - Innendienst (m/w/d)}
% \stelle{Chemiker - Prüfleitung HPLC, LCMS (m/w/d) --- Referenznumer PLHPLC22}
% \stelle{Biologe, Biotechnologe, Biochemiker - Proteinanalytik, GMP (m/w/d)}


%
\ifthenelse {\equal{\english}{\empty}}{\staats{deutsch}}{\staats{german}}
%\strengths{Züverlässig,kommunikativ,Flexibel}{organisiert,autodidaktisch }
\ifthenelse {\equal{\english}{\empty}}{\strengths{analytisch, ausdauernd, konsequent}}{\strengths{analytical, enduring, determined}}


\social[linkedin]{jacob-goeppert}
% \social[linkedin]{sanaz-goeppert-asadollahpour}
% \social[facebook]{face}
% \social[xing]{Sanaz\_GoeppertAsadollahpour}

\signature{Jacob Göppert}

% to show numerical labels in the bibliography (default is to show no labels); only useful if you make citations in your resume
%\makeatletter
%\renewcommand*{\bibliographyitemlabel}{\@biblabel{\arabic{enumiv}}}
%\makeatother
%\renewcommand*{\bibliographyitemlabel}{[\arabic{enumiv}]}% CONSIDER REPLACING THE ABOVE BY THIS

% bibliography with mutiple entries
%\usepackage{multibib}
%\newcites{book,misc}{{Books},{Others}}
%----------------------------------------------------------------------------------
%            content
%----------------------------------------------------------------------------------
\begin{document}


%\stelle{Chemikerin als Manager von Analytik-Dienstleistungen}
% \stelle{Biochemiker- PCR Assays (m/w/d) Referenznummer: DES00220220407}
% \stelle{Biochemiker als Sales Consultant - Innendienst (m/w/d)}
% \stelle{Chemiker - Prüfleitung HPLC, LCMS (m/w/d) --- Referenznumer PLHPLC22}
\stelle{Biologe, Biotechnologe, Biochemiker - Proteinanalytik, GMP (m/w/d)}

% \stelle{Oligonucleotid Produktion}



\newcommand{\firma}{}
\newcommand{\adresse}{}
\newcommand{\anschrift}{}
\newcommand{\anrede}{}
\newcommand{\geschlecht}{}
\newcommand{\kontakt}{none}
\newcommand{\typ}{}
\newcommand{\stelleText}{}


\renewcommand{\typ}{forschung}
% \renewcommand{\typ}{produktion}
% \renewcommand{\typ}{management}




\renewcommand{\kontakt}{ASD}
\renewcommand{\geschlecht}{Herr}



% \collectionnew{kontakt}
% \collectionadd[anrede]{kontakt}{none}
% \collectionadd[name]{kontakt}{none}



\renewcommand{\stelleText}{Biochemikerin für Proteinanalytik}
\renewcommand{\firma}{Phadia GmbH}
\renewcommand{\anschrift}{Munzinger Straße 7\\79111 Freiburg im Breisgau}


\renewcommand{\adresse}{\anschrift}
\renewcommand{\anrede}{Sehr geehrte Damen und Herren}


\ifthenelse{ \equal{\kontakt} {none}}
{ 
    \renewcommand{\adresse}{\anschrift}
    \renewcommand{\anrede}{Sehr geehrte Damen und Herren}
}
{
    \renewcommand{\adresse}{\kontakt \\ \anschrift}
    \ifthenelse{ \equal{\geschlecht} {Herr}}
        {\renewcommand{\anrede}{\ifthenelse{ \equal{\english}{\empty} {Sehr geehrter Herr} {Dear Mr.}}  ~\kontakt}}
        {\renewcommand{\anrede}{\ifthenelse{ \equal{\english}{\empty} {Sehr geehrte Frau}  {Dear Mrs.}} ~\kontakt}}

}
% {
% \renewcommand{\adresse}{\kontakt \\Munzinger Straße 7\\79111 Freiburg im Breisgau}
% \newcommand{\anrede}{Sehr geehrte Damen und Herren}


% 

%\recipient{ChemCon GmbH}{Herr Andreas Kohler\\Engesserstr. 4 b\\79108 Freiburg im Breisgau}
\recipient{JobRad GmbH}{Recruiting Team\\Heinrich-von-Stephan-Straße 13\\79100 Freiburg im Breisgau}
\date{\today}
\opening{Sehr geehrte Damen und Herren,}
\closing{Mit freundlichen Grüßen,}
\enclosure[Anhang]{Lebenslauf}          % use an optional argument to use a string other than "Enclosure", or redefine \enclname
\lettertitle{Bewerbung}
\makelettertitle
\justifying

ich bin derzeit in der Endphase meiner Dissertation auf dem Gebiet der Biochemie, die ich in den nächsten sechs Monaten abschließen werde. Da meine Finanzierung inzwischen ausgelaufen ist, bin ich derzeit arbeitslos gemeldet, und suche bis zu meinem Berufseinstieg in der Industrie einen Minijob. Damit will ich zum einen mein Arbeitslosengeld etwas aufbessern (max. 165 \euro), um anderen wünsche ich mir eine sinnvolle Beschäftigung für ein paar Stunden in der Woche neben dem Feinschliff meiner Dissertation.\par
Da ich im weiteren Verlauf von meiner Karriere Stellen mit Projektverantwortung anstrebe, würde ich jetzt gerne Erfahrung in der Buchhaltung machen.


\makeletterclosing

% }

% 


\recipient{Spindiag GmbH}{Engesserstr. 4 a\\79108 Freiburg im Breisgau}
\date{\today}
\opening{Sehr geehrte Damen und Herren,}
\closing{Mit freundlichen Grüßen}
\enclosure[Anhang]{Lebenslauf, Zeugnisse, Zertifikate}          % use an optional argument to use a string other than "Enclosure", or redefine \enclname
\lettertitle{Bewerbung}
\makelettertitle
\justifying

% {\setlength{\parindent}{0cm}
%Bei der Suche nach meiner ersten Industrieanstellung nach meiner Dissertation bin ich auf ihre Ausschreibung Marketingmanagerin gestoßen. 
% Im Verlauf meiner Doktorarbeit konnte ich meine Begeisterung für hoch-spezifische und maßgeschneiderte Lösungen für komplexe Problemstellungen in der Chemie weiter ausbauen. Bereits in meiner Dissertation war ich stark an konkreten Ergebnissen interessiert, und ich möchte in meiner nächsten beruflichen Tätigkeit diese Produkte nicht nur herstellen, sondern auf den Markt begleiten. Daher reizt mich die ausgeschriebene Stelle als XXX sehr, die diese beiden Aspekte vereint. (Ruf am Institut bezüglich Betriebsklima als KMU?)\par
Hiermit möchte ich mich auf die ausgeschriebene Stelle als Biochemiker für PCR-Assays bei der Spindiag GmbH bewerben. Derzeit befinde ich in der Endphase meiner Dissertation, die ich voraussichtlich im Juli 2022 abschließen werde. Ich arbeite im Bereich der Kristallografie von Proteinen des E.Coli Bakteriums zur Untersuchung des Leigh-Syndroms unter der Leitung von Professor Plattner \& Professor Friedrich an den Instituten für organische Chemie und für Biochemie. Mein Aufgabenbereich reicht dabei von der Anzucht der Baktrien bis zur Auswertung und Aufbereitung der Kristallografiedaten.\par
Meine Arbeit deckt ein breites Spektrum an Gebieten ab und stellt somit eine gute Grundlage für die ausgeschriebene Stelle dar, da ich bereits heute schon für sehr unterschiedliche Fragestellungen Lösungswege erarbeite. Ich bin daher damit vertraut, meine Probleme und Prozesse im Kontext größerer Projekte und Organisationen zu verstehen. Meine Arbeit im universitären Umfeld hat darüber hinaus meine Begeisterung für forschungsnahe Aufgaben weiter gefördert. Daher ist die ausgeschriebene Stelle für mich sehr reizvoll.\par 
% Bereits in meiner Dissertation war ich stark an konkreten Ergebnissen interessiert und ich möchte in meiner nächsten beruflichen Tätigkeit nicht nur Produkte herstellen, sondern diese auch auf den Markt begleiten, weswegen ich die ausgeschriebene Position sehr reizvoll finde.\par
% Daher reizt mich die ausgeschriebene Stelle sehr, da sie diese beiden Aspekte vereint.\par
% Im Verlauf meiner Doktorarbeit konnte ich meine Begeisterung für hochspezifische und maßgeschneiderte Lösungen für komplexe Problemstellungen in der Chemie ausbauen.


% Derzeit bin ich in der Endphase meiner Dissertation, die ich voraussichtlich im Juni 2022 abschließen werde. Ich arbeite im Bereich der Kristallografie von Proteinen unter der Leitung von Prof. Plattner \& Prof. Friedrich an den Instituten für organische Chemie und Biochemie. Dieses breite Spektrum an Arbeitsgebieten stellt eine gute Grundlage für die ausgeschriebene Stelle im Bereich des Marketingmanagements dar, da ich bereits heute schon für sehr unterschiedliche Fragestellungen Lösungswege erarbeite.\par

%Es folgt dem Weg, den ich bereits im Laufe meines Studiums eingeschlagen hatte, in dem ich mich bereits mit angewandter und mit organischer Chemie beschäftigen konnte.\par
% Zu Beginn meiner Dissertation bei Herrn Prof. Plattner ergab sich leider das Problem, dass die eigentlich eingeplanten Proteine für die Kristallografie nicht verfügbar waren. Durch meine Sozialkompetenz und meine Eigeninitiative konnte ich schließlich den Kontakt zu Prof. Friedrich herstellen, der mir die Möglichkeit gab, in seiner Gruppe die benötigten Proteine selbst herzustellen. Dabei hatte ich außerdem die Gelegenheit, mich mit der Unterstützung seiner kompetenten Mitarbeiter eigenständig in ein für mich sehr neues und ungewohntes Gebiet einzuarbeiten.\par
Ein anschauliches Beispiel meiner lösungsorientierten Philosophie ist ein Problem, das sich zu Beginn meiner Dissertation bei Herrn Professor Plattner ergab: Die ursprünglich eingeplanten Proteine für die Kristallografie waren nicht verfügbar. Durch meine Sozialkompetenz und meine Eigeninitiative konnte ich schließlich den Kontakt zu Professor Friedrich herstellen, der mir die Möglichkeit gab, in seiner Gruppe die benötigten Proteine selbst herzustellen. Dabei hatte ich außerdem die Gelegenheit, mich mit der Unterstützung seiner kompetenten Mitarbeiter eigenständig in ein für mich sehr neues und ungewohntes Gebiet einzuarbeiten.\par
% Im Verlauf der Dissertation hatte ich die Gelegenheit, mich tiefgehend mit einer Reihe von Techniken und Methoden der Molekularbiologie und der Großmolekularkristallografie auseinander zu setzen. Besonders hervorzuheben ist dabei XXX, aber auch YYY und ZZZ\par
Im Verlauf der Dissertation konnte ich mich tiefgehend mit einer Reihe von Techniken und Methoden der Molekularbiologie und der Großmolekularkristallografie auseinandersetzen. Besonders hervorzuheben ist dabei meine Erfahrung mit der PCR selbst, insbesondere der ortsspezifischen Mutagenese, sowie verschiedener Vorbereitungs- und Aufbereitungsprozesse wie DNA-Extraktion, Agarose-Gelelektrophorese und Transformation \& Kultivierung von E.coli.
% Mein Interesse reicht auch über den reinen Fachbereich hinaus, und ich bilde mich selbst aktiv weiter. So habe ich mich neben verschiedenen fachspezifischen Fortbildungen wie GMP und Sicherheit in der Gentechnik bereits mit BWL und Projektmanagement beschäftigt.\par
Mein Interesse reicht allerdings über den reinen Fachbereich hinaus und ich bilde mich selbst aktiv weiter. So habe ich mich neben verschiedenen fachspezifischen Fortbildungen wie GMP und Sicherheit in der Gentechnik bereits mit BWL und Projektmanagement beschäftigt. Ich habe außerdem eine hohe IT-Affinität und kenne mich mit unterschiedlichen Software-Tools bestens aus. Für die Arbeit an der Dissertation ist es unausweichlich, sich neben fundiertem chemischem Fachwissen solide Kenntnisse in der Informations- und Datenverarbeitung zu erarbeiten.\par % Auch der Bereich von Social-Media und digitalen Medien interessiert mich sehr, und ich habe darin auch schon viel Erfahrung im privaten Bereich gesammelt.\par

Meine Gehaltsvorstellung liegen bei ca. 65.000~\euro{} und ich werde meine Dissertation voraussichtlich bis Ende Juli 2022 abgeschlossen haben. Ich würde mich sehr freuen, sie bei einem Bewerbungsgespräch persönlich von mir und meinen Fähigkeiten überzeugen zu können. Sollten Sie noch weitere Unterlagen benötigen, reiche ich Ihnen diese gerne nach.\par\vspace{1em}
% \newline

% \begin{itemize}
%     \item Organisation Organik-praktikum -> interdisziplinäre Management-erfahrung, 
%     \item Corona -> Lehre online
%     \item Kooperation Friedrich, eigenständiges einarbeiten in neuen Forschungsbereich
%     \item Selbstständige Problemlösung-> Akquise neuer Kooperationspartner
%     \item Weiterbildungen
%     \begin{itemize}
%         \item BWL
%         \item Management
%     \end{itemize}
%     \item Sicherheit Gentechnik
%     \item GMP

% \end{itemize}

\makeletterclosing

% }



% \pagebreak
%

\recipient{7Bioscience GmbH}{Dekan-Martin-Straße 21\\79395 Neuenburg}
\date{\today}
\opening{Sehr geehrte Damen und Herren}
\closing{Mit freundlichen Grüßen}
\enclosure[Anhang]{Lebenslauf, Zeugnisse, Zertifikate}          % use an optional argument to use a string other than "Enclosure", or redefine \enclname
\lettertitle{Bewerbung}
\makelettertitle
\justifying

% {\setlength{\parindent}{0cm}
%Bei der Suche nach meiner ersten Industrieanstellung nach meiner Dissertation bin ich auf ihre Ausschreibung Marketingmanagerin gestoßen. 
% Im Verlauf meiner Doktorarbeit konnte ich meine Begeisterung für hoch-spezifische und maßgeschneiderte Lösungen für komplexe Problemstellungen in der Chemie weiter ausbauen. Bereits in meiner Dissertation war ich stark an konkreten Ergebnissen interessiert, und ich möchte in meiner nächsten beruflichen Tätigkeit diese Produkte nicht nur herstellen, sondern auf den Markt begleiten. Daher reizt mich die ausgeschriebene Stelle als XXX sehr, die diese beiden Aspekte vereint. (Ruf am Institut bezüglich Betriebsklima als KMU?)\par
Hiermit möchte ich mich auf die ausgeschriebene Stelle als Biochemiker im Vertrieb bei der 7Bioscience GmbH bewerben. Derzeit befinde ich in der Endphase meiner Dissertation, die ich voraussichtlich im Juli 2022 abschließen werde. Ich arbeite im Bereich der Kristallografie von Proteinen des E.Coli Bakteriums unter der Leitung von Professor Plattner \& Professor Friedrich an den Instituten für organische Chemie und für Biochemie. \par
Meine Arbeit deckt ein breites Spektrum an Gebieten ab und stellt daher eine gute Grundlage für die ausgeschriebene Stelle im Bereich des Vertriebs dar. Ich erarbeite bereits heute für sehr unterschiedliche Fragestellungen Lösungswege und beherrsche somit die Sprache verschiedener Fachrichtungen wie der Biologie, der (Bio)Chemie und der Analytik. Bereits in meiner Dissertation war ich stark an konkreten Ergebnissen interessiert und ich möchte in meiner nächsten beruflichen Tätigkeit nicht nur Produkte herstellen, sondern diese auch auf den Markt begleiten, weswegen ich die ausgeschriebene Position sehr reizvoll finde.\par
% Daher reizt mich die ausgeschriebene Stelle sehr, da sie diese beiden Aspekte vereint.\par
% Im Verlauf meiner Doktorarbeit konnte ich meine Begeisterung für hochspezifische und maßgeschneiderte Lösungen für komplexe Problemstellungen in der Chemie ausbauen.


% Derzeit bin ich in der Endphase meiner Dissertation, die ich voraussichtlich im Juli 2022 abschließen werde. Ich arbeite im Bereich der Kristallografie von Proteinen unter der Leitung von Prof. Plattner \& Prof. Friedrich an den Instituten für organische Chemie und Biochemie. Dieses breite Spektrum an Arbeitsgebieten stellt eine gute Grundlage für die ausgeschriebene Stelle im Bereich des Marketingmanagements dar, da ich bereits heute schon für sehr unterschiedliche Fragestellungen Lösungswege erarbeite.\par

%Es folgt dem Weg, den ich bereits im Laufe meines Studiums eingeschlagen hatte, in dem ich mich bereits mit angewandter und mit organischer Chemie beschäftigen konnte.\par
% Zu Beginn meiner Dissertation bei Herrn Prof. Plattner ergab sich leider das Problem, dass die eigentlich eingeplanten Proteine für die Kristallografie nicht verfügbar waren. Durch meine Sozialkompetenz und meine Eigeninitiative konnte ich schließlich den Kontakt zu Prof. Friedrich herstellen, der mir die Möglichkeit gab, in seiner Gruppe die benötigten Proteine selbst herzustellen. Dabei hatte ich außerdem die Gelegenheit, mich mit der Unterstützung seiner kompetenten Mitarbeiter eigenständig in ein für mich sehr neues und ungewohntes Gebiet einzuarbeiten.\par
Ein anschauliches Beispiel meiner lösungsorientierten Philosophie ist ein Problem, das sich zu Beginn meiner Dissertation bei Herrn Professor Plattner ergab: Die ursprünglich eingeplanten Proteine für die Kristallografie waren nicht verfügbar. Durch meine Sozialkompetenz und meine Eigeninitiative konnte ich schließlich den Kontakt zu Professor Friedrich herstellen, der mir die Möglichkeit gab, in seiner Gruppe die benötigten Proteine selbst herzustellen. Dabei hatte ich außerdem die Gelegenheit, mich mit der Unterstützung seiner kompetenten Mitarbeiter eigenständig in ein für mich sehr neues und ungewohntes Gebiet einzuarbeiten.\par
% Im Verlauf der Dissertation hatte ich die Gelegenheit, mich tiefgehend mit einer Reihe von Techniken und Methoden der Molekularbiologie und der Großmolekularkristallografie auseinander zu setzen. Besonders hervorzuheben ist dabei XXX, aber auch YYY und ZZZ\par
Im Verlauf der Dissertation konnte ich mich tiefgehend mit einer Reihe von Techniken und Methoden der Molekularbiologie und der Großmolekularkristallografie auseinandersetzen.
% Besonders hervorzuheben ist dabei XXX, aber auch YYY und ZZZ. 
% Mein Interesse reicht auch über den reinen Fachbereich hinaus, und ich bilde mich selbst aktiv weiter. So habe ich mich neben verschiedenen fachspezifischen Fortbildungen wie GMP und Sicherheit in der Gentechnik bereits mit BWL und Projektmanagement beschäftigt.\par
Mein Interesse reicht allerdings über den reinen Fachbereich hinaus und ich bilde mich selbst aktiv weiter. So habe ich mich neben verschiedenen fachspezifischen Fortbildungen wie GMP und Sicherheit in der Gentechnik bereits mit BWL und Projektmanagement beschäftigt. Ich habe außerdem eine hohe IT- und Onlineaffinität und kenne mich mit unterschiedlichen Software-Tools bestens aus. Für die Arbeit an der Dissertation ist es unausweichlich, sich neben fundiertem chemischem Fachwissen solide Kenntnisse in der Informations- und Datenverarbeitung zu erarbeiten. Auch der Bereich von Social-Media und digitalen Medien interessiert mich sehr, und ich habe darin auch schon viel Erfahrung im privaten Bereich gesammelt.\par

Meine Gehaltsvorstellungen liegen bei ca. 65.000~\euro{} und ich werde meine Dissertation voraussichtlich bis Ende Juli 2022 abgeschlossen haben. Ich würde mich sehr freuen, sie bei einem Bewerbungsgespräch persönlich von mir und meinen Fähigkeiten überzeugen zu können. Sollten Sie noch weitere Unterlagen benötigen, reiche ich Ihnen diese gerne nach.\par\vspace{1em}
% \newline

% \begin{itemize}
%     \item Organisation Organik-praktikum -> interdisziplinäre Management-erfahrung, 
%     \item Corona -> Lehre online
%     \item Kooperation Friedrich, eigenständiges einarbeiten in neuen Forschungsbereich
%     \item Selbstständige Problemlösung-> Akquise neuer Kooperationspartner
%     \item Weiterbildungen
%     \begin{itemize}
%         \item BWL
%         \item Management
%     \end{itemize}
%     \item Sicherheit Gentechnik
%     \item GMP

% \end{itemize}

\makeletterclosing

% }

% \pagebreak
% 

\recipient{menal GmbH}{Im Hausgrün 15 \\ 79312 Emmendingen}
\date{\today}
\opening{Sehr geehrte Damen und Herren,}
\closing{Mit freundlichen Grüßen}
\enclosure[Anhang]{Lebenslauf, Zeugnisse, Zertifikate}          % use an optional argument to use a string other than "Enclosure", or redefine \enclname
\lettertitle{Bewerbung}
\makelettertitle
\justifying

% {\setlength{\parindent}{0cm}
%Bei der Suche nach meiner ersten Industrieanstellung nach meiner Dissertation bin ich auf ihre Ausschreibung Marketingmanagerin gestoßen. 
% Im Verlauf meiner Doktorarbeit konnte ich meine Begeisterung für hoch-spezifische und maßgeschneiderte Lösungen für komplexe Problemstellungen in der Chemie weiter ausbauen. Bereits in meiner Dissertation war ich stark an konkreten Ergebnissen interessiert, und ich möchte in meiner nächsten beruflichen Tätigkeit diese Produkte nicht nur herstellen, sondern auf den Markt begleiten. Daher reizt mich die ausgeschriebene Stelle als XXX sehr, die diese beiden Aspekte vereint. (Ruf am Institut bezüglich Betriebsklima als KMU?)\par
Hiermit möchte ich mich auf die ausgeschriebene Stelle als Chemiker in der Prüfleitung der menal GmbH bewerben. Derzeit befinde ich in der Endphase meiner Dissertation, die ich voraussichtlich im Juli 2022 abschließen werde. Ich arbeite im Bereich der Kristallografie von Proteinen des E.coli Bakteriums zur Untersuchung des Leigh-Syndroms unter der Leitung von Professor Plattner \& Professor Friedrich an den Instituten für organische Chemie und für Biochemie. Mein Aufgabenbereich reicht dabei von der Anzucht der Bakterien bis zur Auswertung und Aufbereitung der Kristallografiedaten.\par
Meine Arbeit deckt ein breites Spektrum an Gebieten ab und stellt daher eine gute Grundlage für die ausgeschriebene Stelle im Bereich der Prüfleitung dar. Ich erarbeite bereits heute für sehr unterschiedliche Fragestellungen Lösungswege und beherrsche somit die Sprache verschiedener Fachrichtungen wie der Biologie, der (Bio)Chemie und der Analytik. Bereits in meiner Dissertation war ich stark an konkreten Ergebnissen interessiert und ich möchte in meiner nächsten beruflichen Tätigkeit nicht nur Produkte und Prozesse entwickeln, sondern diese auch wirtschaftlich produktiv zum Einsatz bringen, weswegen ich die ausgeschriebene Position sehr reizvoll finde.\par
% Daher reizt mich die ausgeschriebene Stelle sehr, da sie diese beiden Aspekte vereint.\par
% Im Verlauf meiner Doktorarbeit konnte ich meine Begeisterung für hochspezifische und maßgeschneiderte Lösungen für komplexe Problemstellungen in der Chemie ausbauen.


% Derzeit bin ich in der Endphase meiner Dissertation, die ich voraussichtlich im Juli 2022 abschließen werde. Ich arbeite im Bereich der Kristallografie von Proteinen unter der Leitung von Prof. Plattner \& Prof. Friedrich an den Instituten für organische Chemie und Biochemie. Dieses breite Spektrum an Arbeitsgebieten stellt eine gute Grundlage für die ausgeschriebene Stelle im Bereich des Marketingmanagements dar, da ich bereits heute schon für sehr unterschiedliche Fragestellungen Lösungswege erarbeite.\par

%Es folgt dem Weg, den ich bereits im Laufe meines Studiums eingeschlagen hatte, in dem ich mich bereits mit angewandter und mit organischer Chemie beschäftigen konnte.\par
% Zu Beginn meiner Dissertation bei Herrn Prof. Plattner ergab sich leider das Problem, dass die eigentlich eingeplanten Proteine für die Kristallografie nicht verfügbar waren. Durch meine Sozialkompetenz und meine Eigeninitiative konnte ich schließlich den Kontakt zu Prof. Friedrich herstellen, der mir die Möglichkeit gab, in seiner Gruppe die benötigten Proteine selbst herzustellen. Dabei hatte ich außerdem die Gelegenheit, mich mit der Unterstützung seiner kompetenten Mitarbeiter eigenständig in ein für mich sehr neues und ungewohntes Gebiet einzuarbeiten.\par
Ein anschauliches Beispiel meiner lösungsorientierten Philosophie ist ein Problem, das sich zu Beginn meiner Dissertation bei Herrn Professor Plattner ergab: Die ursprünglich eingeplanten Proteine für die Kristallografie waren nicht verfügbar. Durch meine Sozialkompetenz und meine Eigeninitiative konnte ich schließlich den Kontakt zu Professor Friedrich herstellen, der mir die Möglichkeit gab, in seiner Gruppe die benötigten Proteine selbst herzustellen. Dabei hatte ich außerdem die Gelegenheit, mich mit der Unterstützung seiner kompetenten Mitarbeiter eigenständig in ein für mich sehr neues und ungewohntes Gebiet einzuarbeiten. \par
% Im Verlauf der Dissertation hatte ich die Gelegenheit, mich tiefgehend mit einer Reihe von Techniken und Methoden der Molekularbiologie und der Großmolekularkristallografie auseinander zu setzen. Besonders hervorzuheben ist dabei XXX, aber auch YYY und ZZZ\par
Im Verlauf der Dissertation konnte ich mich tiefgehend mit einer Reihe von Techniken und Methoden der Molekularbiologie, der Proteinreinigung und der Großmolekularkristallografie auseinandersetzen. Besonders hervorzuheben ist dabei meine Erfahrung mit der Chromatografie mit Äkta, sowie der Bestimmung von Proteinkonzentrationen und der SDS-Page Elektrophorese. %\par
Mein Interesse reicht allerdings über den reinen Fachbereich hinaus und ich bilde mich selbst aktiv weiter. So habe ich mich neben verschiedenen fachspezifischen Fortbildungen wie \emph{GMP} und \emph{Sicherheit in der Gentechnik} bereits mit \emph{BWL} und \emph{Projektmanagement} beschäftigt. Diese Managementkenntnisse konnte ich darüber hinaus auch bei der komplett eigenständigen Planung und Durchführung von Fortgeschrittenenpraktika in meinen Lehrtätigkeiten zum Einsatz bringen und vertiefen. Ich habe außerdem eine hohe IT-Affinität und kenne mich mit unterschiedlichen Software-Tools bestens aus. Für die Arbeit an der Dissertation ist es unausweichlich, sich neben fundiertem chemischem Fachwissen solide Kenntnisse in der Informations- und Datenverarbeitung zu erarbeiten.\par

Ich werde meine Dissertation voraussichtlich bis Ende Juli 2022 abgeschlossen haben. Ich würde mich sehr freuen, Sie bei einem Bewerbungsgespräch persönlich von mir und meinen Fähigkeiten überzeugen zu können. Sollten Sie noch weitere Unterlagen benötigen, reiche ich Ihnen diese gerne nach.\par\vspace{1em}
% \newline

% \begin{itemize}
%     \item Organisation Organik-praktikum -> interdisziplinäre Management-erfahrung, 
%     \item Corona -> Lehre online
%     \item Kooperation Friedrich, eigenständiges einarbeiten in neuen Forschungsbereich
%     \item Selbstständige Problemlösung-> Akquise neuer Kooperationspartner
%     \item Weiterbildungen
%     \begin{itemize}
%         \item BWL
%         \item Management
%     \end{itemize}
%     \item Sicherheit Gentechnik
%     \item GMP

% \end{itemize}

\makeletterclosing

% }

% 

% 


\recipient{Phadia GmbH}{Munzinger Straße 7\\79111 Freiburg im Breisgau}
\date{\today}
\opening{Sehr geehrte Damen und Herren,}
\closing{Mit freundlichen Grüßen}
\enclosure[Anhang]{Lebenslauf, Zeugnisse, Zertifikate}          % use an optional argument to use a string other than "Enclosure", or redefine \enclname
\lettertitle{Bewerbung}
\makelettertitle
\justifying



% {\setlength{\parindent}{0cm}
%Bei der Suche nach meiner ersten Industrieanstellung nach meiner Dissertation bin ich auf ihre Ausschreibung Marketingmanagerin gestoßen. 
% Im Verlauf meiner Doktorarbeit konnte ich meine Begeisterung für hoch-spezifische und maßgeschneiderte Lösungen für komplexe Problemstellungen in der Chemie weiter ausbauen. Bereits in meiner Dissertation war ich stark an konkreten Ergebnissen interessiert, und ich möchte in meiner nächsten beruflichen Tätigkeit diese Produkte nicht nur herstellen, sondern auf den Markt begleiten. Daher reizt mich die ausgeschriebene Stelle als XXX sehr, die diese beiden Aspekte vereint. (Ruf am Institut bezüglich Betriebsklima als KMU?)\par
ich möchte mich auf die ausgeschriebene Stelle als Biochemiker für Proteinanalyse bei der Phadia GmbH bewerben. Derzeit befinde ich in der Endphase meiner Dissertation, die ich voraussichtlich im Juli 2022 abschließen werde. Ich arbeite im Bereich der Kristallografie von Proteinen des E.Coli Bakteriums zur Untersuchung des Leigh-Syndroms unter der Leitung von Professor Plattner \& Professor Friedrich an den Instituten für organische Chemie und für Biochemie. Mein Aufgabenbereich reicht dabei von der Anzucht der Baktrien bis zur Auswertung und Aufbereitung der Kristallografiedaten.\par
Meine Arbeit deckt das gesamte Spektrum der Proteinherstellung ab und stellt somit eine gute Grundlage für die ausgeschriebene Stelle dar. Bereits in meiner Dissertationszeit war ich stark an konkreten Ergebnissen interessiert und ich möchte in meiner nächsten beruflichen Tätigkeit nicht nur Produkte entwickeln, sondern diese auch wirtschaftlich produktiv zum Einsatz bringen. Aus diesem Grund habe ich bereits erfolgreich einen GMP-Kurs absolviert, und finde ich die ausgeschriebene Position sehr reizvoll.\par 
% Bereits in meiner Dissertation war ich stark an konkreten Ergebnissen interessiert und ich möchte in meiner nächsten beruflichen Tätigkeit nicht nur Produkte herstellen, sondern diese auch auf den Markt begleiten, weswegen ich die ausgeschriebene Position sehr reizvoll finde.\par
% Daher reizt mich die ausgeschriebene Stelle sehr, da sie diese beiden Aspekte vereint.\par
% Im Verlauf meiner Doktorarbeit konnte ich meine Begeisterung für hochspezifische und maßgeschneiderte Lösungen für komplexe Problemstellungen in der Chemie ausbauen.


% Derzeit bin ich in der Endphase meiner Dissertation, die ich voraussichtlich im Juni 2022 abschließen werde. Ich arbeite im Bereich der Kristallografie von Proteinen unter der Leitung von Prof. Plattner \& Prof. Friedrich an den Instituten für organische Chemie und Biochemie. Dieses breite Spektrum an Arbeitsgebieten stellt eine gute Grundlage für die ausgeschriebene Stelle im Bereich des Marketingmanagements dar, da ich bereits heute schon für sehr unterschiedliche Fragestellungen Lösungswege erarbeite.\par

%Es folgt dem Weg, den ich bereits im Laufe meines Studiums eingeschlagen hatte, in dem ich mich bereits mit angewandter und mit organischer Chemie beschäftigen konnte.\par
% Zu Beginn meiner Dissertation bei Herrn Prof. Plattner ergab sich leider das Problem, dass die eigentlich eingeplanten Proteine für die Kristallografie nicht verfügbar waren. Durch meine Sozialkompetenz und meine Eigeninitiative konnte ich schließlich den Kontakt zu Prof. Friedrich herstellen, der mir die Möglichkeit gab, in seiner Gruppe die benötigten Proteine selbst herzustellen. Dabei hatte ich außerdem die Gelegenheit, mich mit der Unterstützung seiner kompetenten Mitarbeiter eigenständig in ein für mich sehr neues und ungewohntes Gebiet einzuarbeiten.\par
Ein anschauliches Beispiel meiner lösungsorientierten Philosophie ist ein Problem, das sich zu Beginn meiner Dissertation bei Herrn Professor Plattner ergab: Die ursprünglich eingeplanten Proteine für die Kristallografie waren nicht verfügbar. Durch meine Sozialkompetenz und meine Eigeninitiative konnte ich schließlich den Kontakt zu Professor Friedrich herstellen, der mir die Möglichkeit gab, in seiner Gruppe die benötigten Proteine selbst herzustellen. Dabei hatte ich außerdem die Gelegenheit, mich mit der Unterstützung seiner kompetenten Mitarbeiter eigenständig in ein für mich sehr neues und ungewohntes Gebiet einzuarbeiten.\par
% Im Verlauf der Dissertation hatte ich die Gelegenheit, mich tiefgehend mit einer Reihe von Techniken und Methoden der Molekularbiologie und der Großmolekularkristallografie auseinander zu setzen. Besonders hervorzuheben ist dabei XXX, aber auch YYY und ZZZ\par
Im Verlauf der Dissertation konnte ich mich tiefgehend mit einer Reihe von Techniken und Methoden der Molekularbiologie und der Großmolekularkristallografie auseinandersetzen. Besonders hervorzuheben ist dabei meine Erfahrung mit der PCR selbst, insbesondere der ortsspezifischen Mutagenese, sowie verschiedener Vorbereitungs- und Aufbereitungsprozesse wie DNA-Extraktion, Agarose-Gelelektrophorese und Transformation \& Kultivierung von E.coli.
% Mein Interesse reicht auch über den reinen Fachbereich hinaus, und ich bilde mich selbst aktiv weiter. So habe ich mich neben verschiedenen fachspezifischen Fortbildungen wie GMP und Sicherheit in der Gentechnik bereits mit BWL und Projektmanagement beschäftigt.\par
Mein Interesse reicht allerdings über den reinen Fachbereich hinaus und ich bilde mich selbst aktiv weiter. So habe ich mich neben verschiedenen fachspezifischen Fortbildungen wie den bereits erwähnten GMP und Sicherheit in der Gentechnik bereits mit BWL und Projektmanagement beschäftigt. Ich habe außerdem eine hohe IT-Affinität und kenne mich mit unterschiedlichen Software-Tools bestens aus. Für die Arbeit an der Dissertation ist es unausweichlich, sich neben fundiertem chemischem Fachwissen solide Kenntnisse in der Informations- und Datenverarbeitung zu erarbeiten.\par % Auch der Bereich von Social-Media und digitalen Medien interessiert mich sehr, und ich habe darin auch schon viel Erfahrung im privaten Bereich gesammelt.\par

% Meine G.V.I. liegen bei ca. 65.000~\euro{} und 
% Meine Gehaltsvorstellungen liegen bei ca. 68.000~\euro{} auf Basis einer 40h Woche \`a zwölf Gehältern und 
% \`a
% Ich werde meine Dissertation voraussichtlich bis Ende Juli 2022 abgeschlossen haben.
Ich freue mich darauf, sie bei einem Bewerbungsgespräch persönlich von mir und meinen Fähigkeiten zu überzeugen.\par\vspace{1em}
% Sollten Sie noch weitere Unterlagen benötigen, reiche ich Ihnen diese gerne nach.

% \newline

% \begin{itemize}
%     \item Organisation Organik-praktikum -> interdisziplinäre Management-erfahrung, 
%     \item Corona -> Lehre online
%     \item Kooperation Friedrich, eigenständiges einarbeiten in neuen Forschungsbereich
%     \item Selbstständige Problemlösung-> Akquise neuer Kooperationspartner
%     \item Weiterbildungen
%     \begin{itemize}
%         \item BWL
%         \item Management
%     \end{itemize}
%     \item Sicherheit Gentechnik
%     \item GMP

% \end{itemize}
‌
\makeletterclosing

% }







\recipient{\firma{}}{\adresse{}}
\opening{\anrede{},}
\date{\today}
% \date{3. November 2022}
\closing{Mit freundlichen Grüßen}
\enclosure[Anhang]{Lebenslauf, Zeugnisse, Zertifikate}          % use an optional argument to use a string other than "Enclosure", or redefine \enclname
\lettertitle{Bewerbung}
\makelettertitle
\justifying



% {\setlength{\parindent}{0cm}
%Bei der Suche nach meiner ersten Industrieanstellung nach meiner Dissertation bin ich auf ihre Ausschreibung Marketingmanagerin gestoßen. 
% Im Verlauf meiner Doktorarbeit konnte ich meine Begeisterung für hoch-spezifische und maßgeschneiderte Lösungen für komplexe Problemstellungen in der Chemie weiter ausbauen. Bereits in meiner Dissertation war ich stark an konkreten Ergebnissen interessiert, und ich möchte in meiner nächsten beruflichen Tätigkeit diese Produkte nicht nur herstellen, sondern auf den Markt begleiten. Daher reizt mich die ausgeschriebene Stelle als XXX sehr, die diese beiden Aspekte vereint. (Ruf am Institut bezüglich Betriebsklima als KMU?)\par
ich möchte mich auf die ausgeschriebene Stelle als \stelleText~\firmaText~bewerben. Derzeit befinde ich in der Endphase meiner Dissertation, die ich voraussichtlich Ende 2022 abschließen werde. Ich arbeite im Bereich der Kristallografie von Proteinen des E.Coli Bakteriums zur Untersuchung des Leigh-Syndroms unter der Leitung von Professor Plattner \& Professor Friedrich an den Instituten für organische Chemie und für Biochemie. Mein Aufgabenbereich reicht dabei von der Anzucht der Baktrien bis zur Auswertung und Aufbereitung der Kristallografiedaten.\par
Meine Arbeit deckt das gesamte Spektrum der Proteinherstellung ab und stellt somit eine gute Grundlage für die ausgeschriebene Stelle dar. Bereits in meiner Dissertationszeit war ich stark an konkreten Ergebnissen interessiert und ich möchte in meiner nächsten beruflichen Tätigkeit nicht nur Produkte entwickeln, sondern diese auch wirtschaftlich produktiv zum Einsatz bringen. Aus diesem Grund habe ich bereits erfolgreich einen GMP-Kurs absolviert, und finde ich die ausgeschriebene Position sehr reizvoll.\par 
% Bereits in meiner Dissertation war ich stark an konkreten Ergebnissen interessiert und ich möchte in meiner nächsten beruflichen Tätigkeit nicht nur Produkte herstellen, sondern diese auch auf den Markt begleiten, weswegen ich die ausgeschriebene Position sehr reizvoll finde.\par
% Daher reizt mich die ausgeschriebene Stelle sehr, da sie diese beiden Aspekte vereint.\par
% Im Verlauf meiner Doktorarbeit konnte ich meine Begeisterung für hochspezifische und maßgeschneiderte Lösungen für komplexe Problemstellungen in der Chemie ausbauen.


% Derzeit bin ich in der Endphase meiner Dissertation, die ich voraussichtlich im Juni 2022 abschließen werde. Ich arbeite im Bereich der Kristallografie von Proteinen unter der Leitung von Prof. Plattner \& Prof. Friedrich an den Instituten für organische Chemie und Biochemie. Dieses breite Spektrum an Arbeitsgebieten stellt eine gute Grundlage für die ausgeschriebene Stelle im Bereich des Marketingmanagements dar, da ich bereits heute schon für sehr unterschiedliche Fragestellungen Lösungswege erarbeite.\par

%Es folgt dem Weg, den ich bereits im Laufe meines Studiums eingeschlagen hatte, in dem ich mich bereits mit angewandter und mit organischer Chemie beschäftigen konnte.\par
% Zu Beginn meiner Dissertation bei Herrn Prof. Plattner ergab sich leider das Problem, dass die eigentlich eingeplanten Proteine für die Kristallografie nicht verfügbar waren. Durch meine Sozialkompetenz und meine Eigeninitiative konnte ich schließlich den Kontakt zu Prof. Friedrich herstellen, der mir die Möglichkeit gab, in seiner Gruppe die benötigten Proteine selbst herzustellen. Dabei hatte ich außerdem die Gelegenheit, mich mit der Unterstützung seiner kompetenten Mitarbeiter eigenständig in ein für mich sehr neues und ungewohntes Gebiet einzuarbeiten.\par
Ein anschauliches Beispiel meiner lösungsorientierten Philosophie ist ein Problem, das sich zu Beginn meiner Dissertation bei Herrn Professor Plattner ergab: Die ursprünglich eingeplanten Proteine für die Kristallografie waren nicht verfügbar. Durch meine Sozialkompetenz und meine Eigeninitiative konnte ich schließlich den Kontakt zu Professor Friedrich herstellen, der mir die Möglichkeit gab, in seiner Gruppe die benötigten Proteine selbst herzustellen. Dabei hatte ich außerdem die Gelegenheit, mich mit der Unterstützung seiner kompetenten Mitarbeiter eigenständig in ein für mich sehr neues und ungewohntes Gebiet einzuarbeiten.\par
% Im Verlauf der Dissertation hatte ich die Gelegenheit, mich tiefgehend mit einer Reihe von Techniken und Methoden der Molekularbiologie und der Großmolekularkristallografie auseinander zu setzen. Besonders hervorzuheben ist dabei XXX, aber auch YYY und ZZZ\par
Im Verlauf der Dissertation konnte ich mich tiefgehend mit einer Reihe von Techniken und Methoden der Molekularbiologie und der Großmolekularkristallografie auseinandersetzen. Besonders hervorzuheben ist dabei meine Erfahrung mit der PCR selbst, insbesondere der ortsspezifischen Mutagenese, sowie verschiedener Vorbereitungs- und Aufbereitungsprozesse wie DNA-Extraktion, Agarose-Gelelektrophorese und Transformation \& Kultivierung von E.coli.
% Mein Interesse reicht auch über den reinen Fachbereich hinaus, und ich bilde mich selbst aktiv weiter. So habe ich mich neben verschiedenen fachspezifischen Fortbildungen wie GMP und Sicherheit in der Gentechnik bereits mit BWL und Projektmanagement beschäftigt.\par
Mein Interesse reicht allerdings über den reinen Fachbereich hinaus und ich bilde mich selbst weiter. So habe ich mich neben verschiedenen fachspezifischen Fortbildungen wie den bereits erwähnten GMP und Sicherheit in der Gentechnik bereits mit BWL und Projektmanagement beschäftigt. Ich habe außerdem eine hohe IT-Affinität und kenne mich mit unterschiedlichen Software-Tools bestens aus. Es is unausweichlich, sich während der Dissertation neben fundiertem Fachwissen auch solide Kenntnisse in der Informations- und Datenverarbeitung zu erarbeiten.\par % Auch der Bereich von Social-Media und digitalen Medien interessiert mich sehr, und ich habe darin auch schon viel Erfahrung im privaten Bereich gesammelt.\par

% Meine G.V.I. liegen bei ca. 65.000~\euro{} und 
\ifdefempty{\gehalt}{}{Meine Gehaltsvorstellungen liegen bei ca. \gehalt.000~\euro{} auf Basis einer 40h Woche \`a zwölf Gehältern. }
\ifdefempty{\eintritt}{}{Da ich derzeit noch mit der Ausarbeitung meiner Dissertation ausgelastet bin, kann ich eine neue Stelle erst ab \eintritt{} antreten. }
Ich freue mich darauf, sie bei einem Bewerbungsgespräch persönlich von mir und meinen Fähigkeiten zu überzeugen.\par\vspace{1em}
% Sollten Sie noch weitere Unterlagen benötigen, reiche ich Ihnen diese gerne nach.

% \newline

% \begin{itemize}
%     \item Organisation Organik-praktikum -> interdisziplinäre Management-erfahrung, 
%     \item Corona -> Lehre online
%     \item Kooperation Friedrich, eigenständiges einarbeiten in neuen Forschungsbereich
%     \item Selbstständige Problemlösung-> Akquise neuer Kooperationspartner
%     \item Weiterbildungen
%     \begin{itemize}
%         \item BWL
%         \item Management
%     \end{itemize}
%     \item Sicherheit Gentechnik
%     \item GMP

% \end{itemize}

\makeletterclosing

% }




%\begin{CJK*}{UTF8}{gbsn}                          % to typeset your resume in Chinese using CJK
%-----       resume       ---------------------------------------------------------
\makecvtitle%

% \ifthenelse{ \equal{\typ} {forschung}}
\IfSubStr{\typ}{Forsch}
{   
\def\protein{1}
\def\kristallographie{0}

\section{Berufserfahrung}
%\subsection{Vocational}

\cventry{07/2017 --- heute }{Wissenschaftliche Mitarbeiterin (Promotion)}{Institut für Organische Chemie, Albert-Ludwigs Universität Freiburg}{}{}{}
\cvlistitem{Selbstständige Planung und Organisation des fortgeschrittenen Praktikums in organischer Chemie für Lehramt-Studenten}

\cvitem{}{\bfseries Molekularbiologie}
% \cvlistitem{Ortsspezifische Mutagenese durch PCR}
% \ifthenelse{\equal{\marketing}{1}}{
% \cvlistitem{Agarose-Gelelektrophorese}
% \cvlistitem{DNA-Extraktion}
% \cvlistitem{Transformation und Kultivierung von E.coli}

% }{}
\cvitem{}{\bfseries Proteinreinigung}
\ifthenelse{\equal{\protein}{1}}{
    \cvlistitem{Chromatographie mit Äkta (IMAC, IEX, SEC)}
    \cvlistitem{Bestimmung der Proteinkonzentration }
    \cvlistitem{SDS-Page Elektrophorese}
}{}
% \cvitem{}{\bfseries Proteinkristallographie}
\ifthenelse{\equal{\kristallographie}{1}}{
    \cvitem{}{\bfseries Proteinkristallographie}
    \cvlistitem{Proteinkristallisation}
    \cvlistitem{Kristalle fischen}
    \cvlistitem{Datensätze sammeln}
    \cvlistitem{Datensätze durch CCP4 und COOT prozessieren}
    \cvlistitem{3D-Abbildungen mit Pymol erstellen}
}{
    \cvitem{}{\bfseries Proteinkristallographie --- Auswertung (CCP4, COOT) \& Aufbereitung (Pymol)}

}
\cventry{10/2014 --- 10/2015}{Chemielaborexpertin}{Patan Jame Factory (Textilfabrik), Ghaemshahr, Iran}{}{}{}

\cvlistitem{Test von Färbeprozessen (Farb- \& Stoffkombinationen) für die Produktion}



    
% \def\marketing\1
\section{Bildungsweg}
        \cventry{07/2017 --- heute} {Promotionsstudium in Chemie \slash  ~Biochemie}{Albert-Ludwigs-Universität Freiburg, Deutschland \newline Gruppen von Prof. Plattner \& Prof. Friedrich}{}{}{}
        % \cventry{07/2017 --- heute} {Promotionsstudium in Chemie \slash  ~Biochemie}{Albert-Ludwigs-Universität Freiburg, Deutschland \newline Gruppe von Prof. Plattner}{}{}{}
        \cvlistitem{Lehre und Betreuung des chemischen Praktikums (Medizin)}
        \cvlistitem{Betreuung des chemischen Praktikums Biologie}
        \cvlistitem{Erfahrung im Organisieren von Online-Praktika und Betreuung via Zoom \& ILIAS (univeritätsinterne Plattform für Kollaboration und Datenaustausch )}
        \cventry{6. September 2023}{erfolgreiche Verteidigung der Dissertation, Urkundeübergabe vsstl. 26.10.2023}{}{}{}{}
    \subsection{Studium}
        \cventry{10/2012 --- 09/2014} {Master of Science in Organischer Chemie}{Technische Universität Malek Ashtar Teheran, Iran}{}{}{}
        % \cvlistitem{Schwerpunkt: Organische Chemie}
        \cventry{10/2011 --- 07/2012} {Vorbereitung zur zweiten nationalen Hochschulaufnahmeprüfung (Konkur, Master)}{}{}{}{}
        \cvlistitem{Titel Abschlussarbeit: \emph{Synthesis and Characterization of Thermal and Thermochemical of Coupled Imidazolium-Pyridinium Ionic Liquid}}
        \cventry{10/2007 --- 09/2011} {Bachelor of Science in Angewandter Chemie}{Universität Isfahan, Iran}{}{}{}
        %\cvlistitem{Schwerpunkt XXX}
        %\cvlistitem{Title Abschlussarbeit \emph{XXX}}

    \subsection{Schulbildung}
        \cventry{09/2006 --- 07/2007} {Vorbereitung zur nationalen Hochschulaufnahmeprüfung (Konkur)}{}{}{}{}
        \cventry{10/2005 --- 09/2006} {Abitur}{
        }{}{}{}
        % Beit ol Moghaddas College, Ghaemshahr, Iran}{}{}{}
        % \cventry{10/2005 --- 09/2006} {Mittlere Reife}{"MEHHR"  Highschool, Ghaemshahr, Iran}{}{}{}


        %  (\RL{مهر})


        % \cventry{10/2004 --- 01/2010}{Dipl.-Ing. Mikrosystemtechnik}{Universität Freiburg}{}{\textit{Note 1,8}}{}  % arguments 3 to 6 can be left empty
        % \cvitem{}{Diplomarbeit: \emph{Modellierung der Signalflankenform und Entwurf einer statistischen Timing Analyse digitaler CMOS Schaltungen im Subthresholdbereich}}
        % % \cvitem{Prüfer}{Prof. Dr.-Ing. Y. Manoli, Prof. Dr.-Ing. L. Reindl,}
        % % \cvitem{Note}{1,0}
        % \subsection{Schulbildung}
        % \cventry{09/01--06/04}{Allgemeine Hochschulreife}{IBG -- Technisches Gymnasium}{Lahr}{\textit{Note 1,7}}{}
        % \cventry{09/95--06/01}{Mittlere Reife}{Kooperatives Bildungszentrum -- Realschule}{Seelbach}{\textit{Note 2,7}}{}

}{
    
\section{Berufserfahrung}
%\subsection{Vocational}

\cventry{01/2024 --- heute }{Wissenschaftlicher Mitarbeiter (Postdoc)}{Fakultät für Physiologie, Albert-Ludwigs Universität Freiburg}{}{}{}
\cvlistitem{Internes Projektmanagement im Rahmen des Forschungsclusters \textit{nanodiag BW} und \textit{nEOdiag}
  \begin{itemize}
    \item Einkauf: Design von Ausgangsstoffen und Auswahl von und Kommunikation mit internationalen Lieferanten
    \item Beschaffung von Laborressourcen
    \item Monitoring Projektstatus und Dokumentation
  \end{itemize}
}
\cvlistitem{Eigenständige Forschung am porenformenden Protein \textit{Aerolysin} für die elektrische-optischen Nanopormessung zur Proteinsequenzierung
  \begin{itemize}
    \item Proteinlabeling mit custom unnatural Aminosäuren
    \item Ortspezifische Punktmutation auf DNA und Proteinproduktion \textit{in-vitro} und \textit{in-vivo}
    % \item Proteinproduktion \& -reinigung in-vitro und in-vivo
    % \item manuelle Nanopormessung
  \end{itemize}
}
\cventry{11/2023 --- 12/2023 }{Verkäuferin}{Bäckerei Armbruster}{}{}{}
\cventry{07/2017 --- 10/2023 }{Promotionsstelle}{Fakultät für Chemie und Pharmazie, Albert-Ludwigs Universität Freiburg}{}{}{}
\cvlistitem{Selbstständige Planung \& Organisation des \textit{fortg. Praktikums in org. Chemie Lehramt}}
% \cvlistitem{Lehre und Betreuung des chemischen \textit{Praktikums für Medizin} \& \textit{Praktikums für Biologie}}
% \cvlistitem{Betreuung des chemischen Praktikums für Biologie}
\cvlistitem{Erfahrung im Organisieren von Online-Praktika und Betreuung via Zoom \& ILIAS (univeritätsinterne Plattform für Kollaboration und Datenaustausch )}


\cventry{10/2014 --- 12/2016}{Chemielaborexpertin}{Patan Jame Factory (Textilfabrik), Teheran, Iran}{}{}{}

\cvlistitem{Test von Färbeprozessen (Farb- \& Stoffkombinationen) für die Produktion}



    % \pagebreak
    
% \def\marketing\1

\section{Bildungsweg}

% \cventry{07/2017 --- 10/2023} {Promotionsstudium in Chemie \slash  ~Biochemie}{Albert-Ludwigs-Universität Freiburg, Deutschland \newline Gruppen von Prof. Plattner \& Prof. Friedrich}{}{}{}
\cventry{07/2017 --- 10/2023} {Promotionsstudium in Chemie \slash  ~Biochemie}{Albert-Ludwigs-Universität Freiburg, Deutschland}{}{}{}


\cvitem{}{\bfseries Molekularbiologie}
% \cvlistitem{Ortsspezifische Mutagenese durch PCR}
% \ifthenelse{\equal{\marketing}{1}}{
% \cvlistitem{Agarose-Gelelektrophorese}
% \cvlistitem{DNA-Extraktion}
% \cvlistitem{Transformation und Kultivierung von E.coli}

% }{}
\cvitem{}{\bfseries Proteinreinigung}
% \ifthenelse{\equal{\marketing}{1}}{
% \cvlistitem{Chromatographie mit Äkta (IMAC, IEX, SEC)}
% \cvlistitem{Bestimmung der Proteinkonzentration }
% \cvlistitem{SDS-Page Elektrophorese}
% }{}
% \cvitem{}{\bfseries Proteinkristallographie}
\cvitem{}{\bfseries Proteinkristallographie}
% \cvlistitem{Proteinkristallisation}
% \cvlistitem{Kristalle fischen}
% \cvlistitem{Datensätze sammeln}
% \cvlistitem{Datensätze durch CCP4 und COOT prozessieren}
% \cvlistitem{3D-Abbildungen mit Pymol erstellen}
% \cventry{6. September 2023}{erfolgreiche Verteidigung der Dissertation}{}{}{}{}




\subsection{Studium}
\cventry{10/2012 --- 09/2014} {Master of Science in Organischer Chemie}{Technische Universität Malek Ashtar Teheran, Iran}{}{}{}
% \cvlistitem{Schwerpunkt: Organische Chemie}
% \cvlistitem{Titel Abschlussarbeit: \emph{Thermal and Thermochemical Characterization and Synthesis of Coupled Imidazolium-Pyridinium Ionic Liquid}}
\cvlistitem{Titel Abschlussarbeit: \emph{Synthesis and Thermal and Thermochemical Characterization of Coupled Imidazolium-Pyridinium Ionic Liquid}}
% \cvlistitem{TODO: TECHNIKEN}
\cventry{10/2011 --- 07/2012} {Vorbereitung zur zweiten nationalen Hochschulaufnahmeprüfung (Konkur, Master)}{}{}{}{}
\cventry{10/2007 --- 09/2011} {Bachelor of Science in Angewandter Chemie}{Universität Isfahan, Iran}{}{}{}
%\cvlistitem{Schwerpunkt XXX}
%\cvlistitem{Title Abschlussarbeit \emph{XXX}}

\subsection{Schulbildung}
\cventry{09/2006 --- 07/2007} {Vorbereitung zur ersten nationalen Hochschulaufnahmeprüfung (Konkur, Bachelor)}{}{}{}{}
\cventry{10/2005 --- 09/2006} {Abitur}{
}{Beit ol Moghaddas College, Ghaemshahr, Iran}{}{}{}
% \cventry{10/2005 --- 09/2006} {Mittlere Reife}{"MEHHR"  Highschool, Ghaemshahr, Iran}{}{}{}


%  (\RL{مهر})


% \cventry{10/2004 --- 01/2010}{Dipl.-Ing. Mikrosystemtechnik}{Universität Freiburg}{}{\textit{Note 1,8}}{}  % arguments 3 to 6 can be left empty
% \cvitem{}{Diplomarbeit: \emph{Modellierung der Signalflankenform und Entwurf einer statistischen Timing Analyse digitaler CMOS Schaltungen im Subthresholdbereich}}
% % \cvitem{Prüfer}{Prof. Dr.-Ing. Y. Manoli, Prof. Dr.-Ing. L. Reindl,}
% % \cvitem{Note}{1,0}
% \subsection{Schulbildung}
% \cventry{09/01--06/04}{Allgemeine Hochschulreife}{IBG -- Technisches Gymnasium}{Lahr}{\textit{Note 1,7}}{}
% \cventry{09/95--06/01}{Mittlere Reife}{Kooperatives Bildungszentrum -- Realschule}{Seelbach}{\textit{Note 2,7}}{}

}
% 
\def\protein{1}
\def\kristallographie{0}

\section{Berufserfahrung}
%\subsection{Vocational}

\cventry{07/2017 --- heute }{Wissenschaftliche Mitarbeiterin (Promotion)}{Institut für Organische Chemie, Albert-Ludwigs Universität Freiburg}{}{}{}
\cvlistitem{Selbstständige Planung und Organisation des fortgeschrittenen Praktikums in organischer Chemie für Lehramt-Studenten}

\cvitem{}{\bfseries Molekularbiologie}
% \cvlistitem{Ortsspezifische Mutagenese durch PCR}
% \ifthenelse{\equal{\marketing}{1}}{
% \cvlistitem{Agarose-Gelelektrophorese}
% \cvlistitem{DNA-Extraktion}
% \cvlistitem{Transformation und Kultivierung von E.coli}

% }{}
\cvitem{}{\bfseries Proteinreinigung}
\ifthenelse{\equal{\protein}{1}}{
    \cvlistitem{Chromatographie mit Äkta (IMAC, IEX, SEC)}
    \cvlistitem{Bestimmung der Proteinkonzentration }
    \cvlistitem{SDS-Page Elektrophorese}
}{}
% \cvitem{}{\bfseries Proteinkristallographie}
\ifthenelse{\equal{\kristallographie}{1}}{
    \cvitem{}{\bfseries Proteinkristallographie}
    \cvlistitem{Proteinkristallisation}
    \cvlistitem{Kristalle fischen}
    \cvlistitem{Datensätze sammeln}
    \cvlistitem{Datensätze durch CCP4 und COOT prozessieren}
    \cvlistitem{3D-Abbildungen mit Pymol erstellen}
}{
    \cvitem{}{\bfseries Proteinkristallographie --- Auswertung (CCP4, COOT) \& Aufbereitung (Pymol)}

}
\cventry{10/2014 --- 10/2015}{Chemielaborexpertin}{Patan Jame Factory (Textilfabrik), Ghaemshahr, Iran}{}{}{}

\cvlistitem{Test von Färbeprozessen (Farb- \& Stoffkombinationen) für die Produktion}



    % 
% \def\marketing\1
\section{Bildungsweg}
        \cventry{07/2017 --- heute} {Promotionsstudium in Chemie \slash  ~Biochemie}{Albert-Ludwigs-Universität Freiburg, Deutschland \newline Gruppen von Prof. Plattner \& Prof. Friedrich}{}{}{}
        % \cventry{07/2017 --- heute} {Promotionsstudium in Chemie \slash  ~Biochemie}{Albert-Ludwigs-Universität Freiburg, Deutschland \newline Gruppe von Prof. Plattner}{}{}{}
        \cvlistitem{Lehre und Betreuung des chemischen Praktikums (Medizin)}
        \cvlistitem{Betreuung des chemischen Praktikums Biologie}
        \cvlistitem{Erfahrung im Organisieren von Online-Praktika und Betreuung via Zoom \& ILIAS (univeritätsinterne Plattform für Kollaboration und Datenaustausch )}
        \cventry{6. September 2023}{erfolgreiche Verteidigung der Dissertation, Urkundeübergabe vsstl. 26.10.2023}{}{}{}{}
    \subsection{Studium}
        \cventry{10/2012 --- 09/2014} {Master of Science in Organischer Chemie}{Technische Universität Malek Ashtar Teheran, Iran}{}{}{}
        % \cvlistitem{Schwerpunkt: Organische Chemie}
        \cventry{10/2011 --- 07/2012} {Vorbereitung zur zweiten nationalen Hochschulaufnahmeprüfung (Konkur, Master)}{}{}{}{}
        \cvlistitem{Titel Abschlussarbeit: \emph{Synthesis and Characterization of Thermal and Thermochemical of Coupled Imidazolium-Pyridinium Ionic Liquid}}
        \cventry{10/2007 --- 09/2011} {Bachelor of Science in Angewandter Chemie}{Universität Isfahan, Iran}{}{}{}
        %\cvlistitem{Schwerpunkt XXX}
        %\cvlistitem{Title Abschlussarbeit \emph{XXX}}

    \subsection{Schulbildung}
        \cventry{09/2006 --- 07/2007} {Vorbereitung zur nationalen Hochschulaufnahmeprüfung (Konkur)}{}{}{}{}
        \cventry{10/2005 --- 09/2006} {Abitur}{
        }{}{}{}
        % Beit ol Moghaddas College, Ghaemshahr, Iran}{}{}{}
        % \cventry{10/2005 --- 09/2006} {Mittlere Reife}{"MEHHR"  Highschool, Ghaemshahr, Iran}{}{}{}


        %  (\RL{مهر})


        % \cventry{10/2004 --- 01/2010}{Dipl.-Ing. Mikrosystemtechnik}{Universität Freiburg}{}{\textit{Note 1,8}}{}  % arguments 3 to 6 can be left empty
        % \cvitem{}{Diplomarbeit: \emph{Modellierung der Signalflankenform und Entwurf einer statistischen Timing Analyse digitaler CMOS Schaltungen im Subthresholdbereich}}
        % % \cvitem{Prüfer}{Prof. Dr.-Ing. Y. Manoli, Prof. Dr.-Ing. L. Reindl,}
        % % \cvitem{Note}{1,0}
        % \subsection{Schulbildung}
        % \cventry{09/01--06/04}{Allgemeine Hochschulreife}{IBG -- Technisches Gymnasium}{Lahr}{\textit{Note 1,7}}{}
        % \cventry{09/95--06/01}{Mittlere Reife}{Kooperatives Bildungszentrum -- Realschule}{Seelbach}{\textit{Note 2,7}}{}



\section{Migration nach Deutschland}

\cvitem{10/2015 --- 06/2017}{Deutschkurse in Basel, Schweiz, und Teheran, Iran }
\cvitem{12/2016 --- 06/2017}{Bearbeitung der Einwanderungspapiere in Freiburg, Berlin und Teheran }
\cvitem{12/2016}{Zusage bei Prof. Plattner in Freiburg}
\cvitem{05/2016}{Bewerbungsgespräch bei Prof. Plattner in Freiburg}
\cvitem{10/2015}{Beginn der Promotionsstellensuche in Deutschland}



\section{Kenntnisse \& Fähigkeiten}
% \subsection{Sprachen}
\cvitem{\bfseries Sprachen}{Persisch (Muttersprache) - Deutsch (verhandlungssicher) } %\\ Englisch (erweiterte Kenntnisse)}
% \cvitem{}{Deutsch (verhandlungssicher)}
\cvitem{}{Englisch (gute Kenntnisse in Wort und Schrift)}
% \cvitem{Englisch}{erweiterte Kenntnisse}



\cvitem{\bfseries Software-Tools}{Biorender, Pymol, OriginPro, Igor Pro, ChemDraw, Clone Manager, Bioedit, CCP4, COOT}
\cvitem{\bfseries Labortechniken}{Cloning, Proteindesign, Proteinproduktion, Westernblot, Krystallographie, manuelle Nanopormessung}





\section{Qualifikationen}
\cvitem{\bfseries GMP}{Wochenkurs (03/2021)}
\cvlistitem{Qualitätsmanagementsysteme (CAPA)}
\cvlistitem{Dokumentationspraxis}
\cvlistitem{Risikobewertung und Risikomanagement}
\cvlistitem{Qualifizierung und Validierung}

\vspace{0.8em}

\cvitem{\bfseries Projektmanagment}{2-tägiger Kurs (01/2021)}
% \cvlistitem{SMART Ziele eines Projekts definieren}
% \cvlistitem{GANTT-Charts erstellen}
\cvlistitem{GANTT \& SMART --- klassisches Projektmanagement}
\cvlistitem{SCRUM ---  agiles Projektmanagement}

\vspace{0.8em}
\cvitem{\bfseries Basiswissen BWL}{3-tägiger Kurs (02/2021)}
\cvlistitem{Marketing}
\cvlistitem{Teilbereiche der Betriebswirtschaft}
\cvlistitem{Relevanz betriebswirtschaftlicher Entscheidungsprozesse}
\cvlistitem{Zielsysteme von Organisationen}
\cvlistitem{Marktumfeld}

\vspace{0.8em}
\cvitem{\bfseries Sicherheit in der Gentechnik}{2-tägiger Kurs (10/2019)}
\vspace{-1.475em}
\cvlistitem{Gefährdungspotentiale von Organismen bei gentechnischen Arbeiten}
\cvlistitem{Sicherheitsmaßnahmen für gentechnische Bereiche}
\cvlistitem{Rechtsvorschriften zu Sicherheitsmaßnahmen}







\closingSection{}
% 
\section{Interessen}

{Wandern, Filme, Tanzen}




\makesignature


% \section{Sprachen}
% \cvitemwithcomment{Deutsch}{Mutterspache}{}
% \cvitemwithcomment{Englisch}{verhandlungssicher}{}

% \section{Computerkenntnisse}
% \cvitem{OS}{Mircosoft Windows (95 bis 10), Linux (verschiedene Distributionen)}
% \cvitem{EDV \& Office}{Mircosoft Office (Word, Excel,PowerPoint), \LaTeX}\cvitem{Skriptsprachen}{Python, Java, Matlab, Mathematica}
% \cvitem{IC Digital}{Cadence Enconter, Synopsys Design Compiler, Xilinx ISE \& Vivado}
% \cvitem{IC Analog}{Cadence Virtuoso, ADE ((X)XL), Spectre, AMS, Ocean, Skill, Layouterfahrung}
% \cvitem{Elektronik}{(p)Spice, Target, Eagle, Altium Designer}




% asd
\end{document}
%% end of file `template.tex'.
