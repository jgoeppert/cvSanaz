

\recipient{menal GmbH}{Im Hausgrün 15 \\ 79312 Emmendingen}
\date{\today}
\opening{Sehr geehrte Damen und Herren,}
\closing{Mit freundlichen Grüßen}
\enclosure[Anhang]{Lebenslauf, Zeugnisse, Zertifikate}          % use an optional argument to use a string other than "Enclosure", or redefine \enclname
\lettertitle{Bewerbung}
\makelettertitle
\justifying

% {\setlength{\parindent}{0cm}
%Bei der Suche nach meiner ersten Industrieanstellung nach meiner Dissertation bin ich auf ihre Ausschreibung Marketingmanagerin gestoßen. 
% Im Verlauf meiner Doktorarbeit konnte ich meine Begeisterung für hoch-spezifische und maßgeschneiderte Lösungen für komplexe Problemstellungen in der Chemie weiter ausbauen. Bereits in meiner Dissertation war ich stark an konkreten Ergebnissen interessiert, und ich möchte in meiner nächsten beruflichen Tätigkeit diese Produkte nicht nur herstellen, sondern auf den Markt begleiten. Daher reizt mich die ausgeschriebene Stelle als XXX sehr, die diese beiden Aspekte vereint. (Ruf am Institut bezüglich Betriebsklima als KMU?)\par
Hiermit möchte ich mich auf die ausgeschriebene Stelle als Chemiker in der Prüfleitung der menal GmbH bewerben. Derzeit befinde ich in der Endphase meiner Dissertation, die ich voraussichtlich im Juli 2022 abschließen werde. Ich arbeite im Bereich der Kristallografie von Proteinen des E.coli Bakteriums zur Untersuchung des Leigh-Syndroms unter der Leitung von Professor Plattner \& Professor Friedrich an den Instituten für organische Chemie und für Biochemie. Mein Aufgabenbereich reicht dabei von der Anzucht der Bakterien bis zur Auswertung und Aufbereitung der Kristallografiedaten.\par
Meine Arbeit deckt ein breites Spektrum an Gebieten ab und stellt daher eine gute Grundlage für die ausgeschriebene Stelle im Bereich der Prüfleitung dar. Ich erarbeite bereits heute für sehr unterschiedliche Fragestellungen Lösungswege und beherrsche somit die Sprache verschiedener Fachrichtungen wie der Biologie, der (Bio)Chemie und der Analytik. Bereits in meiner Dissertation war ich stark an konkreten Ergebnissen interessiert und ich möchte in meiner nächsten beruflichen Tätigkeit nicht nur Produkte und Prozesse entwickeln, sondern diese auch wirtschaftlich produktiv zum Einsatz bringen, weswegen ich die ausgeschriebene Position sehr reizvoll finde.\par
% Daher reizt mich die ausgeschriebene Stelle sehr, da sie diese beiden Aspekte vereint.\par
% Im Verlauf meiner Doktorarbeit konnte ich meine Begeisterung für hochspezifische und maßgeschneiderte Lösungen für komplexe Problemstellungen in der Chemie ausbauen.


% Derzeit bin ich in der Endphase meiner Dissertation, die ich voraussichtlich im Juli 2022 abschließen werde. Ich arbeite im Bereich der Kristallografie von Proteinen unter der Leitung von Prof. Plattner \& Prof. Friedrich an den Instituten für organische Chemie und Biochemie. Dieses breite Spektrum an Arbeitsgebieten stellt eine gute Grundlage für die ausgeschriebene Stelle im Bereich des Marketingmanagements dar, da ich bereits heute schon für sehr unterschiedliche Fragestellungen Lösungswege erarbeite.\par

%Es folgt dem Weg, den ich bereits im Laufe meines Studiums eingeschlagen hatte, in dem ich mich bereits mit angewandter und mit organischer Chemie beschäftigen konnte.\par
% Zu Beginn meiner Dissertation bei Herrn Prof. Plattner ergab sich leider das Problem, dass die eigentlich eingeplanten Proteine für die Kristallografie nicht verfügbar waren. Durch meine Sozialkompetenz und meine Eigeninitiative konnte ich schließlich den Kontakt zu Prof. Friedrich herstellen, der mir die Möglichkeit gab, in seiner Gruppe die benötigten Proteine selbst herzustellen. Dabei hatte ich außerdem die Gelegenheit, mich mit der Unterstützung seiner kompetenten Mitarbeiter eigenständig in ein für mich sehr neues und ungewohntes Gebiet einzuarbeiten.\par
Ein anschauliches Beispiel meiner lösungsorientierten Philosophie ist ein Problem, das sich zu Beginn meiner Dissertation bei Herrn Professor Plattner ergab: Die ursprünglich eingeplanten Proteine für die Kristallografie waren nicht verfügbar. Durch meine Sozialkompetenz und meine Eigeninitiative konnte ich schließlich den Kontakt zu Professor Friedrich herstellen, der mir die Möglichkeit gab, in seiner Gruppe die benötigten Proteine selbst herzustellen. Dabei hatte ich außerdem die Gelegenheit, mich mit der Unterstützung seiner kompetenten Mitarbeiter eigenständig in ein für mich sehr neues und ungewohntes Gebiet einzuarbeiten. \par
% Im Verlauf der Dissertation hatte ich die Gelegenheit, mich tiefgehend mit einer Reihe von Techniken und Methoden der Molekularbiologie und der Großmolekularkristallografie auseinander zu setzen. Besonders hervorzuheben ist dabei XXX, aber auch YYY und ZZZ\par
Im Verlauf der Dissertation konnte ich mich tiefgehend mit einer Reihe von Techniken und Methoden der Molekularbiologie, der Proteinreinigung und der Großmolekularkristallografie auseinandersetzen. Besonders hervorzuheben ist dabei meine Erfahrung mit der Chromatografie mit Äkta, sowie der Bestimmung von Proteinkonzentrationen und der SDS-Page Elektrophorese. %\par
Mein Interesse reicht allerdings über den reinen Fachbereich hinaus und ich bilde mich selbst aktiv weiter. So habe ich mich neben verschiedenen fachspezifischen Fortbildungen wie \emph{GMP} und \emph{Sicherheit in der Gentechnik} bereits mit \emph{BWL} und \emph{Projektmanagement} beschäftigt. Diese Managementkenntnisse konnte ich darüber hinaus auch bei der komplett eigenständigen Planung und Durchführung von Fortgeschrittenenpraktika in meinen Lehrtätigkeiten zum Einsatz bringen und vertiefen. Ich habe außerdem eine hohe IT-Affinität und kenne mich mit unterschiedlichen Software-Tools bestens aus. Für die Arbeit an der Dissertation ist es unausweichlich, sich neben fundiertem chemischem Fachwissen solide Kenntnisse in der Informations- und Datenverarbeitung zu erarbeiten.\par

Ich werde meine Dissertation voraussichtlich bis Ende Juli 2022 abgeschlossen haben. Ich würde mich sehr freuen, Sie bei einem Bewerbungsgespräch persönlich von mir und meinen Fähigkeiten überzeugen zu können. Sollten Sie noch weitere Unterlagen benötigen, reiche ich Ihnen diese gerne nach.\par\vspace{1em}
% \newline

% \begin{itemize}
%     \item Organisation Organik-praktikum -> interdisziplinäre Management-erfahrung, 
%     \item Corona -> Lehre online
%     \item Kooperation Friedrich, eigenständiges einarbeiten in neuen Forschungsbereich
%     \item Selbstständige Problemlösung-> Akquise neuer Kooperationspartner
%     \item Weiterbildungen
%     \begin{itemize}
%         \item BWL
%         \item Management
%     \end{itemize}
%     \item Sicherheit Gentechnik
%     \item GMP

% \end{itemize}

\makeletterclosing

% }
